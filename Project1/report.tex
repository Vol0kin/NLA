\documentclass[11pt,a4paper]{article}
\usepackage[english]{babel}					% Use english
\usepackage[utf8]{inputenc}					% Caracteres UTF-8
\usepackage{graphicx}						% Imagenes
\usepackage[hidelinks]{hyperref}			% Poner enlaces sin marcarlos en rojo
\usepackage{fancyhdr}						% Modificar encabezados y pies de pagina
\usepackage{float}							% Insertar figuras
\usepackage[textwidth=390pt]{geometry}		% Anchura de la pagina
\usepackage[nottoc]{tocbibind}				% Referencias (no incluir num pagina indice en Indice)
\usepackage{enumitem}						% Permitir enumerate con distintos simbolos
\usepackage[T1]{fontenc}					% Usar textsc en sections
\usepackage{amsmath}						% Símbolos matemáticos
\usepackage{amsfonts}

% Comando para poner el nombre de la asignatura
\newcommand{\subject}{Numerical Linear Algebra}
\newcommand{\autor}{Vladislav Nikolov Vasilev}
\newcommand{\titulo}{Project 1}
\newcommand{\subtitulo}{Direct methods in optimization with constraints}
\newcommand{\masters}{Master in Fundamental Principles of Data Science}


% Configuracion de encabezados y pies de pagina
\pagestyle{fancy}
\lhead{\autor{}}
\rhead{\subject{}}
\lfoot{\masters}
\cfoot{}
\rfoot{\thepage}
\renewcommand{\headrulewidth}{0.4pt}		% Linea cabeza de pagina
\renewcommand{\footrulewidth}{0.4pt}		% Linea pie de pagina

\begin{document}
\pagenumbering{gobble}

% Title page
\begin{titlepage}
  \begin{minipage}{\textwidth}
    \centering
    \includegraphics[scale=0.25]{img/ub-logo}\\[2cm]
    
    \textsc{\Large \subject\\[0.5cm]}
    \textsc{\uppercase\expandafter{\masters}}\\[1.5cm]
    
    \noindent\rule[-1ex]{\textwidth}{1pt}\\[1.5ex]
    \textsc{{\Huge \titulo\\[0.5ex]}}
    \textsc{{\Large \subtitulo\\}}
    \noindent\rule[-1ex]{\textwidth}{2pt}\\[3.5ex]
  \end{minipage}
  
  \vspace{2cm}
  
  \begin{minipage}{\textwidth}
    \centering
    
    \includegraphics[scale=0.4]{img/ub-ds-logo}
    \vspace{2cm}
    
    \textbf{Author}\\ {\autor{}}\\[2.5ex]
    \textsc{Faculty of Mathematics and Computer Science}\\
    \vspace{1em}
    \textsc{Academic year 2021-2022}
  \end{minipage}
\end{titlepage}

\pagenumbering{arabic}
\tableofcontents
\thispagestyle{empty}				% No usar estilo en la pagina de indice

\newpage

\setlength{\parskip}{1em}

\section{Introduction}

The main goal of this project is to study the basic numerical linear algebra
behind optimization problems. In this case, we are going to consider a convex
optimization problem which has equality and inequality constraints. The goal
is to find a value of $x \in \mathbb{R}^n$ that solves the following
problem:

\begin{equation}
  \label{eq:minimization-problem}
  \begin{aligned}
    \text{minimize } &f(x) = \frac{1}{2}x^TGx + g^Tx \\
    \text{subject to } &A^Tx = b, \quad C^Tx \geq d
  \end{aligned}
\end{equation}

This constrained minimization problem can be solved using Lagrange multipliers.
In order to do so, we have to transform the inequality constraints into equality
constraints. Therefore, we introduce the slack variable $s = C^Tx - d \in \mathbb{R}^m, s \geq 0$.
The Lagrangian is given by the following expression:

\begin{equation}
  \label{eq:lagrangian}
  L(x, \gamma, \lambda, s) = \frac{1}{2}x^TGx + g^Tx - \gamma^T(A^Tx - b) - \lambda^T(C^Tx - d - s)
\end{equation}

The previous expression can be rewritten as:

\begin{equation}
  \label{eq:system}
  \begin{aligned}
    Gx + g - A\gamma - C\lambda &= 0\\
    b - A^T x &= 0 \\
    s + d - C^Tx &= 0 \\
    s_i \lambda_i &= 0, \quad i = 1, \dots, m
  \end{aligned}  
\end{equation}

To solve this problem, we are going to use the Newton's method. Additionally, each step
of the method is going to have two correction substeps that that will help us stay in the feasible
region of the problem.

\noindent \textbf{T1:} In order to solve this problem, let us define $z = (x, \gamma, \lambda, s)$ and
$F: \mathbb{R}^N \rightarrow \mathbb{R}^N$, where $N = n + p + 2m$. The function $F$ can be defined
as follows:

\[
  F(z) = F(x, \gamma, \lambda, s) = (Gx + g - A\gamma - C\lambda, b - A^T x, s + d - C^Tx, s_i \lambda_i)
\]

Our goal is to solve $F(z) = 0$ using Newton's method. This involves computing a Newton step
$\delta_z$ so that for a given point $z_k$ we have that $z_{k+1} = z_k + \delta_z, \forall k \in \mathbb{N}$.
Knowing that $F(z_{k+1}) = F(z_k) + J_F\delta_z$, we can see that this is equivalent to
solving the system $J_F\delta_z = -F(z_k)$, where $J_F$ is the Jacobian matrix.
This matrix is defined as follows:

\[
  J_F =
  \begin{pmatrix}
    \frac{\partial F_1}{\partial x} & \frac{\partial F_1}{\partial \gamma} &
    \frac{\partial F_1}{\partial \lambda} & \frac{\partial F_1}{\partial s} \\
    \frac{\partial F_2}{\partial x} & \frac{\partial F_2}{\partial \gamma} &
    \frac{\partial F_2}{\partial \lambda} & \frac{\partial F_2}{\partial s} \\
    \frac{\partial F_3}{\partial x} & \frac{\partial F_3}{\partial \gamma} &
    \frac{\partial F_3}{\partial \lambda} & \frac{\partial F_3}{\partial s} \\
    \frac{\partial F_4}{\partial x} & \frac{\partial F_4}{\partial \gamma} &
    \frac{\partial F_4}{\partial \lambda} & \frac{\partial F_4}{\partial s}
  \end{pmatrix}
  =
  \begin{pmatrix}
    G & -A & -C & 0 \\
    -A^T & 0 & 0 & 0 \\
    -C^T & 0 & 0 & I \\
    0 & 0 & S & \Lambda
  \end{pmatrix}
  :=
  M_{\text{KKT}}
\]

\noindent where $I$ is a $m \times m$ identity matrix and $S$ and $\Lambda$ are $m \times m$
diagonal matrices containing the values of $s$ and $\lambda$, respectively.

Thus, in order to obtain $\delta_z$ at each step, we have to solve a linear system of equations
defined by the matrix $M_{\text{KKT}}$ and the right hand vector $-F(z_k)$.


\section{Solving the KKT system without inequalities}

The first case that we are going to study is the KKT system without inequalities.
This allows us to represent the matrix $M_{\text{KKT}}$ as a $3 \times 3$ block
matrix:

\[
  M_{\text{KKT}} =
  \begin{pmatrix}
    G & -C & 0 \\
    -C^T & 0 & I \\
    0 & S & \Lambda
  \end{pmatrix}
\]

There are some strategies that all

\subsection{Naive approach}

\subsection{Other strategies}

\subsection{Experimentation}

\section{Solving the general \texttt{KKT} system}

\subsection{$LU$ factorization}

\subsection{$LDL^T$ factorization}

\subsection{Experimentation}

\section{Conclusions}

\section{Additional comments}

\newpage

\begin{thebibliography}{5}

\bibitem{nombre-referencia}
Texto referencia
\\\url{https://url.referencia.com}

\end{thebibliography}

\end{document}

