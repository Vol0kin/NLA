\documentclass[11pt,a4paper]{article}
\usepackage[english]{babel}					% Use english
\usepackage[utf8]{inputenc}					% Caracteres UTF-8
\usepackage{graphicx}						% Imagenes
\usepackage[hidelinks]{hyperref}			% Poner enlaces sin marcarlos en rojo
\usepackage{fancyhdr}						% Modificar encabezados y pies de pagina
\usepackage{float}							% Insertar figuras
\usepackage[textwidth=390pt]{geometry}		% Anchura de la pagina
\usepackage[nottoc]{tocbibind}				% Referencias (no incluir num pagina indice en Indice)
\usepackage{enumitem}						% Permitir enumerate con distintos simbolos
\usepackage[T1]{fontenc}					% Usar textsc en sections
\usepackage{amsmath}						% Símbolos matemáticos
\usepackage{amsfonts}

% Comando para poner el nombre de la asignatura
\newcommand{\subject}{Numerical Linear Algebra}
\newcommand{\autor}{Vladislav Nikolov Vasilev}
\newcommand{\titulo}{Project 2}
\newcommand{\subtitulo}{SVD Applications}
\newcommand{\masters}{Master in Fundamental Principles of Data Science}


% Configuracion de encabezados y pies de pagina
\pagestyle{fancy}
\lhead{\autor{}}
\rhead{\subject{}}
\lfoot{\masters}
\cfoot{}
\rfoot{\thepage}
\renewcommand{\headrulewidth}{0.4pt}		% Linea cabeza de pagina
\renewcommand{\footrulewidth}{0.4pt}		% Linea pie de pagina

\begin{document}
\pagenumbering{gobble}

% Title page
\begin{titlepage}
  \begin{minipage}{\textwidth}
    \centering
    \includegraphics[scale=0.25]{img/ub-logo}\\[2cm]
    
    \textsc{\Large \subject\\[0.5cm]}
    \textsc{\uppercase\expandafter{\masters}}\\[1.5cm]
    
    \noindent\rule[-1ex]{\textwidth}{1pt}\\[1.5ex]
    \textsc{{\Huge \titulo\\[0.5ex]}}
    \textsc{{\Large \subtitulo\\}}
    \noindent\rule[-1ex]{\textwidth}{2pt}\\[3.5ex]
  \end{minipage}
  
  \vspace{2cm}
  
  \begin{minipage}{\textwidth}
    \centering
    
    \includegraphics[scale=0.4]{img/ub-ds-logo}
    \vspace{2cm}
    
    \textbf{Author}\\ {\autor{}}\\[2.5ex]
    \textsc{Faculty of Mathematics and Computer Science}\\
    \vspace{1em}
    \textsc{Academic year 2021-2022}
  \end{minipage}
\end{titlepage}

\pagenumbering{arabic}
\tableofcontents
\thispagestyle{empty}				% No usar estilo en la pagina de indice

\newpage

\setlength{\parskip}{1em}
\setlength{\parindent}{0pt}

\section{Introduction}

The goal of this project is to discuss three common applications of the Singular Value
Decomposition (SVD). First, let's briefly review what the SVD is.

Given a rectangular matrix $A \in \mathbb{R}^{m \times n}$ with $m \geq n$, we can express
it as

\[
  A = U \Sigma V^T
\]

where $U \in \mathbb{R}^{m \times m}$ and $V \in \mathbb{R}^{n \times n}$ are two orthogonal basis
and $\Sigma \in \mathbb{R}^{m \times n}$ is a matrix that can be divided in the diagonal block
$\Sigma\left[1:n, 1:n\right]$ with the singular values the singular values $\sigma_i$ in the diagonal
and the zero block $\Sigma\left[(m-n):m, 1:n\right]$. The singular values are ordered such that
$\sigma_1 \geq \sigma_2 \geq \dots \geq \sigma_n \geq 0$. Since $U$ and $V$ are orthogonal, we have
that $U^{-1} = U^T$ and $V^{-1} = V^T$.

There are some cases in which we can also compute a reduced version of the SVD, which is faster and
reduces the amount of memory needed to store the matrices. This can be particularly useful in scenarios
where the matrix $A$ is rank deficient.

Let $A \in \mathbb{R}^{m \times n}$ be a rectangular matrix with $rank(A) = r$, where $r < n$. For this
case, the reduced SVD can be computed as

\[
  A = U_r \Sigma_r V_r^T
\]

where $U_r \in \mathbb{R}^{m \times r}$ and $V_r^T \in \mathbb{R}^{r \times r}$ are the orthogonal
basis and $\Sigma \in \mathbb{R}^{r \times r}$ is the diagonal matrix containing the nonzero
singular values.

There are many applications of the SVD, but in this project we are going to focus on three of them:
solving the Least Squares Problem, graphic compression and Principal Component Analysis.

\section{Least Squares Problem}

The first application that we are going to address is the Least Squares Problem (LSP).

\subsection{Polynomial fitting}

\subsection{The rank deficient LSP}

\section{Graphics compression}

\section{Principal Component Analysis}

\subsection{Example problem}

\subsection{Genes problem}

\newpage

\begin{thebibliography}{5}

\bibitem{nombre-referencia}
Texto referencia
\\\url{https://url.referencia.com}

\end{thebibliography}

\end{document}

